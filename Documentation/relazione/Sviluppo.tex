% -------------------------------- SVILUPPO -------------------------------------

\chapter{Sviluppo}
\section{Testing automatizzato}

Per il seguente progetto software è stato largamente sfruttato il testing automatizzato, dal momento che è stato scelto un approccio TDD(test-driven-development). Il nostro gruppo ha considerato opportuno testare prevalentemente la componente di logica(model) oltre ai vari controller.
Abbiamo inoltre attribuito importanza alla "pulizia" dei test, per garantire la loro manutenibilità, effettuando i seguenti accorgimenti : 

\begin{itemize}
	\item Utilizzo di una struttura fissa secondo il quale in primis si preparano i dati del test, successivamente si opera su di essi ed infine si controlla che i risultati siano quelli previsti. 
	\item Tentativo di mantenere i test indipendenti, in modo da poter individuare gli errori in modo preciso ed analitico.
	\item Dichiarazione dei metodi di test con nomi autoesplicativi, in modo da poter individuare lo scopo del test senza la necessità di inserire commenti ridondanti e voluminosi.
\end{itemize}

\subsection*{Alessandro Pioggia}

\begin{itemize}
	\item GameEntityTest;
	\item ObstacleTest;
	\item SpeedVector2DTest;
	\item Dimension2DTest;
\end{itemize}

\subsection*{Leon Baiocchi}

\begin{itemize}
    \item SpriteContainerTest;
	\item GameEnvironmentTest;
\end{itemize}

\subsection*{Federico Brunelli}

\begin{itemize}
	\item todo : scrivere le classi di test utilizzate
\end{itemize}

\subsection*{Luca Rengo}

\begin{itemize}
	\item \textsf{\small CharactersTest}
	\item \textsf{\small MapTest}
	\item \textsf{\small SaveTest}
	\item \textsf{\small SecureDataTest}
\end{itemize}



\section{Metodologia di lavoro}

Nella prima fase di realizzazione del progetto ci siamo dedicati all'analisi, in cui abbiamo lavorato in maniera coordinata per definire la struttura e le funzionalità del progetto.
Successivamente siamo passati al design, in cui è stato deciso di creare uno scheletro in UML che descrivesse a grandi linee le entità principali necessarie per il funzionamento, definendo già le dipendenze fra esse. Il processo descritto ci ha permesso di definire una suddivisione del lavoro in 4 parti, in modo da garantire la prosecuzione del lavoro individuale con la fase di design dettagliato. \\
Per facilitare lo sviluppo abbiamo sfruttato un real-time-mantained UML, ovvero una repository in cui ciascun membro, una volta terminata una sessione di lavoro, aveva la premura di aggiornare uno schema UML, costruendo/modificando un diagramma delle classi che descrivesse il lavoro svolto. \\
Per quanto riguarda il DVCS abbiamo optato per l'utilizzo di git, in accordo con le nozioni apprese a lezione. La metodologia utilizzata è stata la seguente:

\begin{itemize}
	\item Sviluppo in feature-branches, ovvero branches in cui venivano realizzate singolarmente le feature dell'applicativo;
	\item Ognuno di noi aveva a disposizione un numero definito di branches indipendenti dagli altri;
	\item I feature-branches venivano poi confluiti nel main-branch.
\end{itemize}

\subsection*{Alessandro Pioggia}
Lavoro svolto : 
\begin{itemize}
	\item Realizzazione del menù principale con relativi reindirizzamenti.
	\item Creazione dello scheletro per l'implementazione delle entità di gioco fisiche, ovvero la creazione dell'interfaccia PhysicalObject la relativa implementazione;
	\item Implementazione di oggetti di gioco pickable (Item) e di ostacoli(Obstacles), realizzazione di sprites compresa;
	\item Creazione di commons, quali Dimension2D e SpeedVector2D;
	\item Creazione dello scheletro PhysicalObjectSprite, per la creazione delle sprite (aspetto di view);
	\item Gestione dei suoni.
\end{itemize}


\subsection*{Leon Baiocchi}

\begin{itemize}
	\item Sistema di gestione delle entità di gioco e delle sprites(EntityContainer, SpriteContainer);
	\item Realizzazione HUD;
	\item Creazione e gestione di effetti;
	\item Modellazione e generazione dell'ambiente di gioco;
	\item Scheletro per la gestione degli eventi e sviluppo di un sistema di rilevazione collisioni(CollisionEventChecker) interno all'ambiente di gioco.
	\item Scheletro per creazione e gestione di comandi in input.
	\item Scheletro per gestione di entità di gioco e sprites(Container, AbstractContainer).
	\item Sviluppo del game engine.
\end{itemize}

\subsection*{Federico Brunelli}

\begin{itemize}
	\item  todo : scrivere quello che si è fatto nel progetto.
\end{itemize}

\subsection*{Luca Rengo}

\begin{itemize}
	\item  \textsf{\small Modellazione del \emph{giocatore} e dei \emph{nemici}.}
	\item \textsf{\small Composizione di una classe astratta per la formazione di una generica scena di gioco.}
	\item \textsf{\small Generazione della Mappa di gioco e delle relative entità dal punto di vista della \emph{View} come le \emph{piattaforme}, \emph{monete}, il \emph{giocatore} e i \emph{nemici}.}
	\item \textsf{\small Implementazione dell'animazione del player.}
	\item \textsf{\small Creazione di una classe per il salvataggio dei dati e delle statistiche di gioco.}
\end{itemize}

\section{Note di sviluppo}

\subsection*{Alessandro}

\begin{itemize}
	\item  \textbf{\textit{JavaFx}} : utilizzato per la realizzazione del menù;
	\item \textbf{\textit{Optionals}} : per evitare di ritornare valori null;
	\item \textbf{\textit{Apache library}} : libreria che ho importato per l'utilizzo delle classi Pair preimplementate;
	\item \textbf{\textit{Gradle}} : strumento che ho utilizzato per importare le varie componenti di javafx;
	\item \textbf{\textit{Reflection}} : utilizzata per la creazione delle statistiche, perchè richiesta da javafx;
	\item \textbf{\textit{JUnit}} : utilizzata per il testing;
	\item \textbf{\textit{Streams e lambda}} : utilizzate dove possibile, per migliorare la leggibilità e l'efficienza.
\end{itemize}

\subsection*{Leon Baiocchi}

\begin{itemize}
	\item \textbf{\textit{JavaFX}}: utilizzato per realizzare l'HUD e gestire le varie sprites.
	\item \textbf{\textit{Optionals}}: utili per evitare NullPointerException.
	\item \textbf{\textit{Apache Commons}}: libreria importata per accedere alle classi MutablePair/ImmutablePair.
	\item \textbf{\textit{Gradle}}: strumento usato per importare e lavorare con tutte le varie dipendenze.
	\item \textbf{\textit{JUnit}}: per fare unit testing.
	\item \textbf{\textit{Lambda e Streams}}: utilizzo lambda e stream per rendere il codice più efficente.
	\item \textbf{\textit{Repository references}}: \href{https://bitbucket.org/aricci303/2021-game-prog-basics/src/master/}{game-programming-basics}
	repository di riferimento durante le fasi iniziali del progetto.
\end{itemize}

\subsection*{Federico Brunelli}

\begin{itemize}
	\item  todo : scrivere quello che si è fatto nel progetto.
\end{itemize}

\subsection*{Luca Rengo}

\begin{itemize}
	\item \textsf{\small \textbf{JavaFX}: sfruttato per la realizzazione della mappa e delle sue componenti.}
	\item \textsf{\small \textbf{Factory}: adoperata per raggruppare le mie classi ed instanziarle più facilmente.}
	\item \textsf{\small \textbf{JUnit}: beneficiata per l'esaminazione delle parti di codice create.}
	\item \textsf{\small \textbf{Lambda}: utilizzate per semplificare alcune parti del programma.}
	\item \textsf{\small \textbf{Gradle}: impiegato per importare i vari moduli di JavaFX, JUnit e le varie dipendenze e per tenere il progetto ben organizzato.}
	\item \textsf{\small \textbf{Apache library}: utilizzata per usufruire delle classi dei Pair.}
	\item \textsf{\small \textbf{json-simple}: adoperata per salvare i file in formato .json}
	\item \textsf{\small \textbf{JavaFX Internazionalization}: usata per la traduzione del gioco in diverse lingue.}
	\item \textsf{\small \textbf{Optionals}: usati quando avevo valori opzionali.}
	\item \textsf{\small \textbf{javax.crypto, java.security}: per encriptare e decriptare i files.}
	\item \textsf{\small \textbf{Regex} : utilizzato per separare riga per riga nel caricamento dei dati dei livelli.}
\end{itemize}