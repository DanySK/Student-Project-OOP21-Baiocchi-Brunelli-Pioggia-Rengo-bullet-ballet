% ------------------------ COMMENTI FINALI---------------------------------------

\chapter{Commenti finali}

In quest'ultimo capitolo si tirano le somme del lavoro svolto e si delineano eventuali sviluppi
futuri.

\textit{Nessuna delle informazioni incluse in questo capitolo verrà utilizzata per formulare la valutazione finale}, a meno che non sia assente o manchino delle sezioni obbligatorie.
%
Al fine di evitare pregiudizi involontari, l'intero capitolo verrà letto dai docenti solo dopo aver formulato la valutazione.

\section{Autovalutazione e lavori futuri}

% ------------------------ ALESSANDRO PIOGGIA ----------------------------------

\subsection*{Alessandro Pioggia}

Il seguente progetto mi ha permesso di crescere tanto e sopratutto di riconoscere e individuare gli errori da me commessi.
A mio avviso si è rivelato fondamentale seguire le direttive a noi impartite, specialmente in un lavoro a gruppi, anche un solo metodo non chiamato in maniera adeguata può mettere in seria difficoltà i compagni.
Devo ammettere che non è stato facile gestire le tempistiche, non ho considerato il tempo necessario per il setup del progetto e per eventuali errori(esempio : errori nella gradle build).
\\

\begin{flushleft}
	Punti di forza:
\end{flushleft}

\begin{itemize}
	\item flessibilità, capacità di mettersi in discussione
	\item comunicazione con i compagni
\end{itemize}

\begin{flushleft}
	Punti deboli :
\end{flushleft}

\begin{itemize}
	\item gestione delle tempistiche
	\item mancato uso di programmazione funzionale
	\item ridotta complessità generale
\end{itemize}

% ------------------------ LEON BAIOCCHI ---------------------------------------

\subsection*{Leon Baiocchi}

%TODO: scrivere qualcosa qui.
\textsf{\small }

\begin{flushleft}
	
\textsf{\small Punti di forza:}\\

\begin{itemize}
	\item \textsf{\small } %TODO: Punti di forza
	\item \textsf{\small }
	\item \textsf{\small }
\end{itemize}

\textsf{\small Punti deboli: }\\

\begin{itemize}
	\item \textsf{\small } %TODO: Punti deboli
	\item \textsf{\small }
	\item \textsf{\small }
\end{itemize}

\end{flushleft}


% ------------------------ FEDERICO BRUNELLI -----------------------------------

\subsection*{Federico Brunelli}

%TODO: scrivere qualcosa qui.
\textsf{\small }

\begin{flushleft}
	
	\textsf{\small Punti di forza:}\\
	
	\begin{itemize}
		\item \textsf{\small } %TODO: Punti di forza
		\item \textsf{\small }
		\item \textsf{\small }
	\end{itemize}
	
	\textsf{\small Punti deboli: }\\
	
	\begin{itemize}
		\item \textsf{\small } %TODO: Punti deboli
		\item \textsf{\small }
		\item \textsf{\small }
	\end{itemize}
	
\end{flushleft}

% ------------------------ LUCA RENGO ------------------------------------------

\subsection*{Luca Rengo}

\textsf{\small Questo primo progetto mi ha sicuramente fatto comprendere meglio l'importanza di una buona organizzazione, gestione delle tempistiche e di quanto sia essenziale una corretta coordinazione e comunicazione del lavoro.}\\
\textsf{\small Alla fine di tutto, ciò che è fondamentale per una produttiva esecuzione del progetto è un effettivo teamwork e workflow. } %determinante

\begin{flushleft}
	
	\textsf{\small Punti di forza:}\\
	
	\begin{itemize}
		\item \textsf{\small Esaminare e discutere i vari aspetti del progetto assieme ai colleghi e cercare di trovare soluzioni comuni sul come affrontarli. }
		\item \textsf{\small Complementarietà del team, ogni membro ha la sua parte specifica del progetto.}
		\item \textsf{\small Risoluzione dei problemi in maniera unita.}
		\item \textsf{\small Scambio di opinioni oneste.}
	\end{itemize}
	
	\textsf{\small Punti deboli: }\\
	
	\begin{itemize}
		\item \textsf{\small Gestione del tempo a disposizione.}
		\item \textsf{\small Poca chiarezza sul da farsi in modo pratico e preciso, dovuto anche al fatto che fosse la prima volta che facevamo un progetto così grande.}
		\item \textsf{\small Realizzazione del lavoro e della sua comunicazione.}
	\end{itemize}
	
\end{flushleft}

\begin{comment}
\textbf{È richiesta una sezione per ciascun membro del gruppo, obbligatoriamente}.
%
Ciascuno dovrà autovalutare il proprio lavoro, elencando i punti di forza e di debolezza in quanto prodotto.
Si dovrà anche cercare di descrivere \emph{in modo quanto più obiettivo possibile} il proprio ruolo all'interno del gruppo.
Si ricorda, a tal proposito, che ciascuno studente è responsabile solo della propria sezione: non è un problema se ci sono opinioni contrastanti, a patto che rispecchino effettivamente l'opinione di chi le scrive.
Nel caso in cui si pensasse di portare avanti il progetto, ad esempio perché effettivamente impiegato, o perché sufficientemente ben riuscito da poter esser usato come dimostrazione di esser capaci progettisti, si descriva brevemente verso che direzione portarlo.
\end{comment}

\section{Difficoltà incontrate e commenti per i docenti}

Questa sezione, \textbf{opzionale}, può essere utilizzata per segnalare ai docenti eventuali problemi o difficoltà incontrate nel corso o nello svolgimento del progetto, può essere vista come una seconda possibilità di valutare il corso (dopo quella offerta dalle rilevazioni della didattica) avendo anche conoscenza delle modalità e delle difficoltà collegate all'esame, cosa impossibile da fare usando le valutazioni in aula per ovvie ragioni.
%
È possibile che alcuni dei commenti forniti vengano utilizzati per migliorare il corso in futuro: sebbene non andrà a vostro beneficio, potreste fare un favore ai vostri futuri colleghi.
%
Ovviamente \textit{il contenuto della sezione non impatterà il voto finale}.